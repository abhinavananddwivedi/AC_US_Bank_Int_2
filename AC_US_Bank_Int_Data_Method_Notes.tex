\documentclass[11pt,]{article}
\usepackage{lmodern}
\usepackage{amssymb,amsmath}
\usepackage{ifxetex,ifluatex}
\usepackage{fixltx2e} % provides \textsubscript
\ifnum 0\ifxetex 1\fi\ifluatex 1\fi=0 % if pdftex
  \usepackage[T1]{fontenc}
  \usepackage[utf8]{inputenc}
\else % if luatex or xelatex
  \ifxetex
    \usepackage{mathspec}
  \else
    \usepackage{fontspec}
  \fi
  \defaultfontfeatures{Ligatures=TeX,Scale=MatchLowercase}
\fi
% use upquote if available, for straight quotes in verbatim environments
\IfFileExists{upquote.sty}{\usepackage{upquote}}{}
% use microtype if available
\IfFileExists{microtype.sty}{%
\usepackage{microtype}
\UseMicrotypeSet[protrusion]{basicmath} % disable protrusion for tt fonts
}{}
\usepackage[margin = 1.5in]{geometry}
\usepackage{hyperref}
\PassOptionsToPackage{usenames,dvipsnames}{color} % color is loaded by hyperref
\hypersetup{unicode=true,
            pdftitle={Integration Among US Banks: Data and Methodology Guide},
            pdfauthor={Abhinav Anand and John Cotter},
            colorlinks=true,
            linkcolor=blue,
            citecolor=magenta,
            urlcolor=red,
            breaklinks=true}
\urlstyle{same}  % don't use monospace font for urls
\usepackage{longtable,booktabs}
\usepackage{graphicx,grffile}
\makeatletter
\def\maxwidth{\ifdim\Gin@nat@width>\linewidth\linewidth\else\Gin@nat@width\fi}
\def\maxheight{\ifdim\Gin@nat@height>\textheight\textheight\else\Gin@nat@height\fi}
\makeatother
% Scale images if necessary, so that they will not overflow the page
% margins by default, and it is still possible to overwrite the defaults
% using explicit options in \includegraphics[width, height, ...]{}
\setkeys{Gin}{width=\maxwidth,height=\maxheight,keepaspectratio}
\IfFileExists{parskip.sty}{%
\usepackage{parskip}
}{% else
\setlength{\parindent}{0pt}
\setlength{\parskip}{6pt plus 2pt minus 1pt}
}
\setlength{\emergencystretch}{3em}  % prevent overfull lines
\providecommand{\tightlist}{%
  \setlength{\itemsep}{0pt}\setlength{\parskip}{0pt}}
\setcounter{secnumdepth}{0}
% Redefines (sub)paragraphs to behave more like sections
\ifx\paragraph\undefined\else
\let\oldparagraph\paragraph
\renewcommand{\paragraph}[1]{\oldparagraph{#1}\mbox{}}
\fi
\ifx\subparagraph\undefined\else
\let\oldsubparagraph\subparagraph
\renewcommand{\subparagraph}[1]{\oldsubparagraph{#1}\mbox{}}
\fi

%%% Use protect on footnotes to avoid problems with footnotes in titles
\let\rmarkdownfootnote\footnote%
\def\footnote{\protect\rmarkdownfootnote}

%%% Change title format to be more compact
\usepackage{titling}

% Create subtitle command for use in maketitle
\newcommand{\subtitle}[1]{
  \posttitle{
    \begin{center}\large#1\end{center}
    }
}

\setlength{\droptitle}{-2em}

  \title{Integration Among US Banks: Data and Methodology Guide}
    \pretitle{\vspace{\droptitle}\centering\huge}
  \posttitle{\par}
    \author{Abhinav Anand and John Cotter}
    \preauthor{\centering\large\emph}
  \postauthor{\par}
      \predate{\centering\large\emph}
  \postdate{\par}
    \date{2018/09/03}

\linespread{1.25}
\usepackage{amsmath}

\begin{document}
\maketitle

\section{Data}\label{data}

The starting period for our sample is January 1 1993 and the ending
period is December 31 2016.

Our sample consists of all US depository credit institutions and bank
holding companies for which data are available in both CRSP and
Compustat. We use the WRDS interface to collect data from both sources.

\subsection{Sample Construction}\label{sample-construction}

Finance and insurance related entries are distributed between SIC codes
6000--6799. The following table illustrates this classification in more
detail.\footnote{The full list can be found on the link:
  \url{http://www.ehso.com/siccodes.php}}

\begin{longtable}[]{@{}ll@{}}
\caption{SIC codes for finance and related industry
groups}\tabularnewline
\toprule
SIC Code & Industry\tabularnewline
\midrule
\endfirsthead
\toprule
SIC Code & Industry\tabularnewline
\midrule
\endhead
6012 & Pay Day Lenders\tabularnewline
6021 & National Commercial Banks\tabularnewline
6022 & State Commercial Banks\tabularnewline
6029 & Commercial Banks, NEC\tabularnewline
6035 & Savings Institution, Federally Chartered\tabularnewline
6036 & Savings Institutions, Not Federally Chartered\tabularnewline
6099 & Functions Related To Depository Banking, NEC\tabularnewline
6111 & Federal \& Federally Sponsored Credit Agencies\tabularnewline
6141 & Personal Credit Institutions\tabularnewline
6153 & Short-Term Business Credit Institutions\tabularnewline
6159 & Miscellaneous Business Credit Institution\tabularnewline
6162 & Mortgage Bankers \& Loan Correspondents\tabularnewline
6163 & Loan Brokers\tabularnewline
6172 & Finance Lessors\tabularnewline
6189 & Asset-Backed Securities\tabularnewline
6199 & Finance Services\tabularnewline
6200 & Security \& Commodity Brokers, Dealers, Exchanges \&
Services\tabularnewline
6211 & Security Brokers, Dealers \& Flotation Companies\tabularnewline
6221 & Commodity Contracts Brokers \& Dealers\tabularnewline
6282 & Investment Advice\tabularnewline
6311 & Life Insurance\tabularnewline
6321 & Accident \& Health Insurance\tabularnewline
6324 & Hospital \& Medical Service Plans\tabularnewline
6331 & Fire, Marine \& Casualty Insurance\tabularnewline
6351 & Surety Insurance\tabularnewline
6361 & Title Insurance\tabularnewline
6399 & Insurance Carriers, NEC\tabularnewline
6411 & Insurance Agents, Brokers \& Service\tabularnewline
6500 & Real Estate\tabularnewline
6510 & Real Estate Operators (No Developers) \& Lessors\tabularnewline
6512 & Operators of Nonresidential Buildings\tabularnewline
6513 & Operators of Apartment Buildings\tabularnewline
6519 & Lessors of Real Property, NEC\tabularnewline
6531 & Real Estate Agents \& Managers (For Others)\tabularnewline
6532 & Real Estate Dealers (For Their Own Account)\tabularnewline
6552 & Land Subdividers \& Developers (No Cemeteries)\tabularnewline
6770 & Blank Checks\tabularnewline
6792 & Oil Royalty Traders\tabularnewline
6794 & Patent Owners \& Lessors\tabularnewline
6795 & Mineral Royalty Traders\tabularnewline
6798 & Real Estate Investment Trusts\tabularnewline
6799 & Investors, NEC\tabularnewline
\bottomrule
\end{longtable}

We collect daily price and return data from the CRSP database for all
entities whose SIC codes lie between 6020--6079 or from 6710--6712
between the above mentioned dates. Commercial banks lie between SIC
codes 6020--6029, saving institutions between 6030--6039, credit unions
between 6060--6069; and bank holding companies between 6710--6712. The
SIC code ranges \(\{6020,\hdots,6079\}\cup \{6710, 6711, 6712\}\) are
referred to henceforth as `admissible' SICs.

We include only common stocks corresponding to codes 10 and 11 and
remove all American Depository Receipts (ADRs) and firms incorporated in
non-US countries. We further delete all entities with nominal stock
prices less than \$1. For firms whose SIC classification changes from
admissible to inadmissible or vice versa, we include them only for the
time duration corresponding to their status as admissible banks. Since
we include all such banks irrespective of whether they are alive or not,
our study is free from survivorship bias.

Finally we include only those US banks whose total assets in 2016 are at
least \$1 billion according to data collected from Compustat. This
leaves us with a final sample of 349 unique banks.

Our attention on public banks with primary listings in the US excludes
several multinational banking corporations which might have secondary
listings in the US but primary listings elsewhere. For example, the
British bank HSBC has a secondary listing on the New York Stock Exchange
but under our definition, we do not include it in the list of US banks.
In the same way, financial service providers such as mutual funds,
insurance companies etc. are not included in our definition of banks.
Since the focus of our paper is to isolate and study integration
dynamics of US banks, inclusion of European or Asian banks with
secondary listings in the US may bias our estimates.

\section{Methodology}\label{methodology}

\subsection{Frequency of Estimation of Bank
Integration}\label{frequency-of-estimation-of-bank-integration}

Our sample stretches from January 1 1993 to December 31 2016. We
partition the duration of the study into quarters and compute bank
integration each quarter from daily bank returns. Hence the integration
estimates start from 1993 Quarter 1 to 2016 Quarter 4---a total of 96
quarters.

Under this setup, there are between 62--66 daily observations for each
bank's return each quarter. We compute the covariance matrix of the 338
US banks each quarter and extract as many principal components each
quarter as are necessary to explain 90\% of bank returns. For banks
which do not contain data for the entire sample period, we start
estimating their integration levels from the time their data begin
appearing in CRSP.

\subsection{Construction of Principal
Components}\label{construction-of-principal-components}

Principal components each quarter are constructed from the covariance
matrices of all banks with available returns in a particular quarter.
For example, at the beginning of the sample there are only 43 banks with
available returns and hence the size of the covariance matrix from which
principal components are extracted is \(43\times 43\). As time
progresses, the coverage of banks increases steadily so that by the end
of the sample (quarter 96) there are 234 admissible banks and hence the
corresponding covariance matrix has dimensions \(234\times 234\).

In constructing covariance matrices, we first remove any bank which has
no available returns for the entire quarter. Further we also remove
banks for which stale returns and missing values exceed a threshold of
30\% quarterly observations.\footnote{Since there are 62--66
  observations each quarter, this means that if a bank in question has
  greater than 22 missing and/or stale entries, we remove it from the
  construction of the covariance matrix.} This takes care of most of the
banks in our sample. For the remnant few banks with leftover missing
values, we replace them with their banks' respective quarterly medians
and then compute the covariance matrix.

\subsubsection{Out-of-Sample Principal Component
Construction}\label{out-of-sample-principal-component-construction}

Once eigenvectors of the covariance matrices are computed in order of
largest to smallest eigenvalue, out-of-sample principal components are
estimated by applying them to observed returns for the subsequent
quarter. For example, eigenvectors from the covariance matrix in 1993Q1
are applied to the covariance matrix in 1993Q2 to obtain principal
components in 1993Q2. Hence there are overall 95 such quarterly
principal component computations---from 1993Q2 to 2016Q4.

Each quarter we collect as many principal components as necessary for
explaining 90\% of returns. Hence the number of principal components
used differs from quarter to quarter.


\end{document}
